\documentstyle[11pt]{article}
%
% Copyright (c) 1995-1996 by Alex Aiken.  All rights reserved.
% Permission is granted to modify and distribute this document for
% for non-commercial purposes, so long as this copyright notice is retained
% in all copies.
%
% Side margins:
% Actual margin is 1 in + this number
\oddsidemargin -0.25in
\evensidemargin -0.25in

% Text width:
\textwidth 6.9in

% Top margin:
% Actual margin is 1.5 in + this number
\topmargin -.3in

% Text height:
\textheight 8.7in

% generally useful macros for writing

% These allow switching interline spacing; the change takes effect immediately:

\makeatletter
\newcommand{\singlespacing}{\let\CS=\@currsize\renewcommand{\baselinestretch}{1}\tiny\CS}
\newcommand{\oneandahalfspacing}{\let\CS=\@currsize\renewcommand{\baselinestretch}{1.25}\tiny\CS}
\newcommand{\doublespacing}{\let\CS=\@currsize\renewcommand{\baselinestretch}{1.5}\tiny\CS}
%setspacingto sets the interline spacing to the value of its argument
% e.g., \setspacingto{1.5} is the same as \doublespacing
\newcommand{\setspacingto}[1]{\let\CS=\@currsize\renewcommand{\baselinestretch}{#1}\tiny\CS}
\makeatother

% nonumber
\newcommand{\nn}{\nonumber}

% Tab for hand-formatting:
\newcommand{\tab}{\hspace*{2em}}

% Angle brackets:
\newcommand{\la}{\langle}
\newcommand{\ra}{\rangle}

% s.t.
\newcommand{\st}{\mbox{\ s.t.\ }}

% otherwise
\newcommand{\ow}{\m{\rm otherwise}}

% if
\newcommand{\mif}{\m{\rm if\ }}

% Harpoons
\newcommand{\rh}{\rightharpoonup}
\newcommand{\lh}{\leftharpoonup}

% Denotational-semantics-style brackets and bottom:
\newcommand{\lbk}{\lbrack\!\lbrack}
\newcommand{\rbk}{\rbrack\!\rbrack}
\newcommand{\bottom}{\perp}

% Projection operator
\newcommand{\proj}{\!\downarrow\!}

% macros for mbox combined with another style
% (useful for changing typefaces in math mode)
\newcommand {\mboxbf}[1]{\mbox{{\bf #1}}}
\newcommand {\mboxit}[1]{\mbox{{\it #1}}}
\newcommand {\mboxem}[1]{\mbox{{\em #1}}}
\newcommand {\m}{\mbox}
\newcommand {\ch}{\rm}
	
% macro for creating a binary operator
%
% example:  \makebinop{\makebinop{\mybmod}{mod}
%	(duplicates the \bmod macro)
%
\def\makebinop#1#2{\def#1{\mskip-\medmuskip \mskip5mu
\mathbin{\rm #2} \penalty900 \mskip5mu \mskip-\medmuskip}}

% Environments for theorems, lemmas, etc.
\newtheorem{theorem}{Theorem}[section]
\newtheorem{lemma}[theorem]{Lemma}
\newtheorem{corollary}[theorem]{Corollary}
\newtheorem{definition}[theorem]{Definition}
\newtheorem{observation}[theorem]{Observation}
\newtheorem{fact}{Fact}[theorem]
\newtheorem{proposition}[theorem]{Proposition}
\newtheorem{example}[theorem]{Example}
\newtheorem{constraint}[theorem]{Constraint}
\newtheorem{axiom}[theorem]{Axiom}
\newtheorem{law}[theorem]{Law}
\newtheorem{algorithm}[theorem]{Algorithm}
\newtheorem{invariant}[theorem]{Invariant}

% Definition of the proof-environment:
\newenvironment{proof}{{\bf Proof:}\quad}{$\Box$}


%
% An environment for formatting programs. 
%
%	written by Hal Perkins
%	adapted by Charles Elkan	9/3/86
%	keyword macros by Anne Neirynck
%
% Usage :
%
% \begin{program}
% program text\\
% program text
% \end{program}
%
% The program environment is a tabbing environment with ten tab stops spaced
% evenly from the left of the page.  Initially the left margin is the second
% tab stop.  Use \+ to indent following lines one more tab stop, \- to undo
% the effect of \+, and \> at the beginning of a line to indent an extra tab.

\newlength{\pgmtab}          %  \pgmtab is the width of each tab in the
\setlength{\pgmtab}{2em}     %  program environment

% boxed program is like program, only boxed!
% This is useful for centering programs and preventing page breaks in programs.
% argument t or b is required.

\newenvironment{boxed-program}[1]{\begin{minipage}[#1]{9in}
\begin{tabbing}\hspace{\pgmtab}\=\hspace{\pgmtab}\=%
\hspace{\pgmtab}\=\hspace{\pgmtab}\=\hspace{\pgmtab}\=\hspace{\pgmtab}\=%
\hspace{\pgmtab}\=\hspace{\pgmtab}\=\hspace{\pgmtab}\=\hspace{\pgmtab}\=%
\hspace{\pgmtab}\=%
\kill}{\end{tabbing}\end{minipage}}

\newenvironment{program}{\begin{tabbing}\hspace{\pgmtab}\=\hspace{\pgmtab}\=%
\hspace{\pgmtab}\=\hspace{\pgmtab}\=\hspace{\pgmtab}\=\hspace{\pgmtab}\=%
\hspace{\pgmtab}\=\hspace{\pgmtab}\=\hspace{\pgmtab}\=\hspace{\pgmtab}\=%
\hspace{\pgmtab}\=%
\+\+\kill}{\end{tabbing}}

% The following commands should be used OUTSIDE math mode

\newcommand {\FUNCTION}{{\bf function\ }}
\newcommand {\BEGIN}{{\bf begin\ }}
\newcommand {\END}{{\bf end}}
\newcommand {\CASE}{{\bf case}}
\newcommand {\OF}{{\bf of}}
\newcommand {\SELECT}{{\bf select\ }}
\newcommand {\WHERE}{{\bf where\ }}
\newcommand {\DECLARE}{{\bf declare\ }}
\newcommand {\ARRAY}{{\bf array\ }}
\newcommand {\LET}{{\bf\ let\ }}
\newcommand {\IN}{{\bf\ in\ }}
\newcommand {\IF}{{\bf if\ }}
\newcommand {\THEN}{{\bf then\ }}
\newcommand {\ELSE}{{\bf else\ }}
\newcommand {\SKIP}{{\bf skip\ }}
\newcommand {\DO}{{\bf do\ }}
% OLD---don't use \OD
\newcommand {\OD}{{\bf od\ }}
\newcommand {\BY}{{\bf by\ }}
\newcommand {\LOOP}{{\bf loop\ }}
\newcommand {\WHILE}{{\bf while\ }}
\newcommand {\TO}{{\bf to\ }}
\newcommand {\DOWNTO}{{\bf down to\ }}
\newcommand {\FOR}{{\bf for\ }}
\newcommand {\FOREACH}{{\bf for each\ }}
\newcommand {\RETURN}{{\bf return\ }}
\newcommand {\REPEAT}{{\bf repeat\ }}
\newcommand {\UNTIL}{{\bf until\ }}
\newcommand {\LCOM}{$(\ast \;$}
\newcommand {\RCOM}{$\ast)$}
\newcommand {\GOTO}{{\bf goto\ }}

% The following commands are for use INSIDE math mode

\newcommand {\OP}[1]{\mbox{\sc #1}}
\newcommand {\op}[1]{\mbox{\sc #1}}
\newcommand {\mm}[1]{\mbox{\rm #1}\;}
\newcommand {\id}[1]{\mbox{\it #1\ }}

\newcommand {\ASSIGN}{\leftarrow }
\newcommand {\MIN}{\OP{min} }
\newcommand {\MOD}{\; {\bf{\rm mod}} \; } 
\newcommand {\LAMBDA}{{\bf \lambda\ }}
\newcommand {\FALSE}{{\em FALSE\ }}

% Miscellaneous notation

\newcommand {\And}{\wedge}
\newcommand {\Or}{\vee}

\newcommand {\thus}{{\dot{. \: .}\;}}

\newcommand {\bigO}[1]{{\cal O}(#1)}

\newcommand {\app}{\!\!:\!}
\newcommand {\hastype}{::}
\newcommand{\hast}{:}
\newcommand{\qt}[1]{\mbox{``#1''}}
\newcommand{\TexComment}[1]{}

\newcommand{\dq}{\m{\tt "}}
\newcommand{\flatqt}[1]{\m{\dq #1 \dq}}
\newcommand{\seq}{\subseteq}
\newcommand{\derives}{\vdash}

% for writing grammars
\newcommand{\grammar}{::=}
\newcommand{\gor}{\,|\,}

% macros for writing inference rules
\newcommand{\infrule}[2]{\displaystyle{\displaystyle\strut{#1}} \over %
                                        {\displaystyle\strut {#2}}}
\newcommand{\cinfrule}[3]{\parbox{14cm}{\hfil$\infrule{#1}{#2}$\hfil}\parbox{4cm}{$\,#3$\hfil}}


\newcommand{\kw}[1]{\m{\bf #1}}
\newcommand{\nt}[1]{\m{\it #1}}
\newcommand{\tm}[1]{\m{#1}}
\newcommand{\s}{\;\,}
\newcommand{\assgn}{\s\m{\tt <-}\s}
\newcommand{\styp}{{\tt SELF\_TYPE}}


\newcommand{\attr}[3]{#1:#2\leftarrow#3}
\newcommand{\classmap}[2]{class(#1) = (#2)}
\newcommand{\tcrule}[3]{\frac{#1}{#2}\eqno\mbox{#3}}
\newcommand{\ossimple}[6]{#1,#2,#3\vdash #4 : #5,#6}
\newcommand{\osrule}[8]{\frac{#7}{\ossimple{#1}{#2}{#3}{#4}{#5}{#6}}\eqno
\mbox{#8}}
\def\U#1{{\sf{}#1}}
\def\S#1{{\tt{}#1}} % NB: we often use \verb+...+ for this also
\def\C#1{{\tt{}#1}}

\begin{document}

\title{CoolAid: The Cool Reference Manual\thanks{
Copyright \copyright 1995-1996 by Alex Aiken.
All rights reserved.}}  

\author{ }
\date{ }
\maketitle



\section{Introduction}

This manual describes the programming language Cool: the {\em C}lassroom
{\em O}bject-{\em O}riented {\em L}anguage.
Cool is a small language that can be
implemented with reasonable effort in a one semester course.  Still, Cool
retains many of the features of modern programming languages including
objects, static typing, and automatic memory management.

Cool programs are sets of {\em classes}.  A class encapsulates the variables
and procedures of a data type.  Instances of a class are
{\em objects}.  In Cool, classes and types are identified; i.e., every
class defines a type.  Classes permit programmers to define new
types and associated procedures (or {\em methods}) specific to those
types.  Inheritance allows new types to extend the behavior of
existing types.

Cool is an {\em expression} language.  Most Cool constructs are
expressions, and every expression has a value and a type.  Cool is
{\em type safe}: procedures are guaranteed to be applied to data of
the correct type.  While static typing imposes a strong discipline on
programming in Cool, it guarantees that no runtime type errors can
arise in the execution of Cool programs.

This manual is divided into informal and formal components.  For a short,
informal overview, the first 12 pages (through Section~\ref{sec-main}) suffices.
The formal description begins with Section~\ref{lex-struct}.

\section{Getting Started}
\label{sec-start}

The reader who wants to get a sense for Cool at the outset should
begin by reading and running the example programs in the directory
{\tt \~{ }cs164/examples}.  Cool source files have extension {\tt .cl}
and Cool assembly files have extension {\tt .s}.
The Cool compiler is {\tt \~{ }cs164/bin/coolc}.  To compile a program:
\begin{verbatim}
coolc [ -o fileout ] file1.cl file2.cl ... filen.cl 
\end{verbatim}

The compiler compiles the files {\tt file1.cl} through {\tt filen.cl}
as if they were concatenated together.  Each file must define a
set of complete classes---class definitions may not be split across
files.  The {\tt -o} option specifies an optional name to use for the
output assembly code.  If {\tt fileout} is not supplied, the output
assembly is named {\tt file1.s}.

The {\tt coolc} compiler generates MIPS assembly code.  Because not
all of the machines the course is using are MIPS-based, Cool
programs are run on a MIPS simulator called {\tt spim}.  To run a cool
program, type
\begin{verbatim}
% spim
(spim) load "file.s"
(spim) run
\end{verbatim}
To run a different program during the same {\tt spim} session, it is
necessary to reinitialize the state of the simulator before loading the
new assembly file:
\begin{verbatim}
(spim) reinit
\end{verbatim}
An alternative---and faster---way to invoke {\tt spim} is with a file:
\begin{verbatim}
spim -file file.s
\end{verbatim}
This form loads the file, runs the program, and exits {\tt spim} when 
the program terminates.
Be sure that {\tt spim} is invoked using the script {\tt \~{ }cs164/bin/spim}.
There is another version of {\tt spim} installed in {\tt /usr/local/bin} on
some systems, but it will not execute Cool programs.  An easy way to be sure
of getting the correct version is to add {\tt \~{ }cs164/bin} to your search
path (before {\tt /usr/local/bin}!).  The {\tt spim} manual is available
on the course Web page.

The following is a complete transcript of the compilation and
execution of {\tt \~{ }cs164/examples/list.cl}.  This program is very
silly, but it does serve to illustrate many of the features of
Cool.

\begin{verbatim}
% coolc list.cl
% spim
SPIM Version 5.4 of Jan. 17, 1994
Copyright 1990-1994 by James R. Larus (larus@cs.wisc.edu).
All Rights Reserved.
See the file README a full copyright notice.
Loaded: /home/n/cs164/lib/trap.handler
(spim) load "list.s"
(spim) run
How many numbers to sort?5
0
1
2
3
4
COOL program successfully executed
(spim) exit
% 
\end{verbatim}

\section{Classes}

All code in Cool is organized into classes.  Each class definition must be
contained in a single source file, but multiple classes may be defined in the
same file.  Class definitions have the form:

\begin{verbatim}
class <type> [ inherits <type> ] is
    <feature_list>
end;
\end{verbatim}

The notation {\tt [ \ldots ]} denotes an optional construct.
All class names are globally visible.  Class names begin with an
uppercase letter.  Classes may not be redefined.

\subsection{Features}

The body of a class definition consists of a list of feature
definitions.  A feature is either an {\em attribute} or a {\em
method}.  An attribute of class {\tt A} specifies a variable that is
part of the state of objects of class {\tt A}.  A method of class {\tt
A} is a procedure that may manipulate the variables and objects of class {\tt A}.

One of the major themes of modern programming languages is {\em
information hiding}, which is the idea that certain aspects of a data
type's implementation should be abstract and hidden from users of the
data type.  Cool supports information hiding through a simple
mechanism: all attributes have scope local to the class, and all
methods have global scope.  Thus, the only way to provide access to
object state in Cool is through methods.

Feature names must begin with a lowercase letter.  No method name
may be defined multiple times in a class, and no attribute name
may be defined multiple times in a class, but a method and an attribute
may have the same name.

A fragment from {\tt list.cl} illustrates simple cases of both attributes
and methods:

\begin{verbatim}
class Cons inherits List is
	xcar : Int;
	xcdr : List;

	isNil() : Bool is false end;

	init(hd : Int, tl : List) : Cons is
	  begin
	    xcar <- hd;
	    xcdr <- tl;
	    self;
	  end
...
end;
\end{verbatim}
In this example, the class {\tt Cons} has two attributes {\tt xcar} and
{\tt xcdr} and two methods {\tt isNil} and {\tt init}.  Note that the types
of attributes, as well as the types of formal parameters and return types
of methods, are explicitly declared by the programmer.  

Given objects {\tt c} of class {\tt Cons} and {\tt l} of class {\tt
List}, we can set the {\tt xcar} and {\tt xcdr} fields by using the
method {\tt init}:

\begin{verbatim}
c.init(1,l)
\end{verbatim}

This notation is {\em object-oriented dispatch}.  There may be many
definitions of {\tt init} methods in many different classes.  The
dispatch looks up the class of the object {\tt c} to decide which {\tt
init} method to invoke.  Because the class of {\tt c} is {\tt Cons},
the {\tt init} method in the {\tt Cons} class is invoked.  Within the
invocation, the variables {\tt xcar} and {\tt xcdr} refer to {\tt c}'s
attributes.  The special variable {\tt self} refers to the object on
which the method was dispatched, which, in the example, is {\tt c} itself.

There is a special form {\tt new C} that generates a fresh object of
class {\tt C}.  An object can be thought of as a record that has a
slot for each of the attributes of the class as well as
pointers to the methods of the class.
A typical dispatch for the {\tt init} method is:
\begin{verbatim}
(new Cons).init(1,new Nil)
\end{verbatim}
This example creates a new cons cell and initializes the ``car'' of
the cons cell to be {\tt 1} and the ``cdr'' to be {\tt new
Nil}.\footnote{In this example, {\tt Nil} is assumed to be a subtype
of {\tt List}.} There is no mechanism in Cool for programmers to
deallocate objects.  Cool has {\em automatic memory management};
objects that cannot be used by the program are deallocated by a
runtime garbage collector.


Attributes are discussed further in Section~\ref{sec-attr}
and methods are discussed further in Section~\ref{sec-method}.


\subsection{Inheritance}
\label{sec-inherit}

If a class definition has the form
\begin{verbatim}
class A inherits B is ... end;
\end{verbatim}
then class {\tt A} inherits the features of {\tt B}.
In this case {\tt B} is the {\em parent} class of {\tt A}
and {\tt A} is a {\em child} class of {\tt B}.

The semantics of {\tt A inherits B} is that {\tt A} has all of the
features defined in {\tt B} in addition to its own features.
In the case that a parent and child both define the same method
name, then the definition given in the child class takes precedence.
It is illegal to redefine attribute names.  Furthermore, for type safety,
it is necessary to place some restrictions on how methods may be redefined
(see Section~\ref{sec-method}).

There is a distinguished class {\tt Object}.  If a class definition
does not specify a parent class, then the class inherits from {\tt
Object} by default.  A class may inherit only from a single class;
this is aptly called ``single inheritance.''\footnote{Some
object-oriented languages allow a class to inherit from multiple
classes, which is equally aptly called ``multiple inheritance.''} The
parent-child relation on classes defines a graph.  This graph may not
contain cycles.  For example, if {\tt A} inherits from {\tt B}, then
{\tt B} must not inherit from {\tt A}.  Furthermore, if {\tt A}
inherits from {\tt B}, then {\tt B} must have a class definition
somewhere in the program.  Because Cool has single inheritance,
it follows that if both of these restrictions are
satisfied, then the inheritance graph forms a tree with {\tt Object}
as the root.

In addition to {\tt Object}, Cool has four other {\em basic classes}: 
	{\tt Int}, {\tt String}, {\tt Bool}, and {\tt IO}.
The basic classes are discussed in Section~\ref{sec-basic}.

\section{Types}

In Cool, every class name is also a type.  In addition, there is a
type \styp\ that can be used in special circumstances.

A {\em type declaration} has the form {\tt x:C}, where {\tt x} is a variable
and {\tt C} is a type.  Every variable must have a type declaration at the
point it is introduced, whether that is in a {\tt let}, {\tt case}, or as
the formal parameter of a method.  The types of all attributes must also be
declared.

The basic type rule in Cool is that if a method or variable expects a value
of type {\tt A}, then any value of type {\tt B} may be used instead, provided
that {\tt A} is an ancestor of {\tt B} in the class hierarchy.  In other
words, if {\tt B} inherits from {\tt A}, either directly or indirectly, then
a {\tt B} can be used wherever an {\tt A} would suffice.  

When an object of class {\tt B} may be used in place of an object of class
{\tt A}, we say that {\tt B} {\em conforms} to {\tt A} or that
$\tt B \leq A$.  As discussed above,
conformance is defined in terms of the inheritance graph.
\begin{definition}[Conformance]
\label{def-conforms}
\rm
Let $\tt A,B,$ and $\tt C$ be types.
\begin{itemize}
\item $\tt A \leq A$ for all types {\tt A}

\item if {\tt A} inherits from {\tt B}, then $\tt A \leq B$

\item if $\tt A \leq B$ and $\tt B \leq C$ then $\tt A \leq C$

\end{itemize}
\end{definition}

Because {\tt Object} is the root of the class hierarchy, it follows that
$\tt A \leq Object$ for all types $\tt A$.

\subsection{SELF\_TYPE}

The type \styp\ is used to refer to the type of the {\tt self}
variable.  This is useful in classes that will be inherited by other
classes, because it allows the programmer to avoid specifying
a fixed final type at the time the class is written.  For example,
the program
\begin{verbatim}
class Silly is
   copy() : SELF_TYPE is self end;
end;

class Sally inherits Silly is end;

class Main is
   x : Sally <- (new Sally).copy();

   main() : Sally is x end;
end;
\end{verbatim}
Because \styp\ is used in the definition of the {\tt copy} method,
we know that the result of {\tt copy} is  the same as the 
type of the {\tt self}
parameter.  Thus, it follows that {\tt (new Sally).copy()} has
type {\tt Sally}, which conforms to the declaration of attribute {\tt x}.

Note that the meaning of \styp\ is not fixed, but depends on the class
in which it is used.  In general, \styp\ may refer to the class {\tt C}
in which it appears, or any class that conforms to {\tt C}.  When it is
useful to make explicit what $\styp$ may refer too, we use the 
name of the class {\tt C} in which \styp\ appears as an index $\styp_{\tt C}$.
This subscript notation is not part of Cool syntax---it is used merely
to make clear in what class a particular occurrence of \styp\ appears.

From Definition~\ref{def-conforms}, it follows that
$\styp_{\tt X} \leq \styp_{\tt X}$.
There is also a special conformance rule for \styp:
\[ \tt \styp_C \leq B \;\m{\rm\ if\ }\; C \leq B \]

Finally, \styp\ may be used in the following places: {\tt new \styp},
as the return type of a method, as the declared type of a {\tt let}
variable, or as the declared type of an attribute. No other uses of
\styp\ are permitted.

\subsection{Type Checking}

The Cool type system guarantees at compile time that execution of a program
cannot result in runtime type errors.  Using the type declarations for
identifiers supplied by the programmer, the type checker infers a type for
every expression in the program.  

It is important to distinguish between the type assigned by the type
checker to an expression at compile time, which we shall call the {\em
static} type of the expression, and the type(s) to which the
expression may evaluate during execution, which we shall call the {\em
dynamic} types.

The distinction between static and dynamic types is needed because the
type checker cannot, at compile time, have perfect information about
what values will be computed at runtime.  Thus, in general, the static and
dynamic types may be different.  What we require, however, is that the
type checker's static types be {\em sound} with respect to the dynamic types.

\begin{definition}
\rm
For any expression  {\tt e}, let $\tt D_e$ be a dynamic type of {\tt e}
and let $\tt S_e$ be the static type inferred by the type checker.
Then the type checker is {\em sound} if for all expressions {\tt e} it is the
case that $\tt D_e \leq S_e$.
\end{definition}

Put another way, we require that the type checker err on the side of
overestimating the type of an expression in those cases where perfect
accuracy is not possible.  Such a type checker will never accept a
program that contains type errors.  However, the price paid is that
the type checker will reject some programs that would actually execute
without runtime errors.

\section{Attributes}
\label{sec-attr}

An attribute definition has the form
\begin{verbatim}
<id> : <type> [ <- <expr> ];
\end{verbatim}
The expression is optional initialization that is executed when a new
object is created.  The static type of the expression must conform to
the declared type of the attribute.  If no initialization is supplied,
then the default initialization is used (see below).

When a new object of a class is created, all of
the inherited and local attributes must be initialized.  Inherited
attributes are initialized first in inheritance order beginning with the
attributes of the greatest ancestor class. Within a given class,
attributes are initialized in the order they appear in the source
text.

 Attributes are local to the class in which they are defined or inherited.
Inherited attributes cannot be redefined.

\subsection{Void}

All variables in Cool are initialized to contain values of the
appropriate type.  The special value {\tt void} is a member of all
types and is used as the default initialization for variables where
no initialization is supplied by the user.

There is a special form {\tt isvoid expr} that tests whether a value
is {\tt void} (see Section~\ref{sec-isvoid}).  In addition, {\tt void}
may be tested for equality.  Void may be passed as an
argument, assigned to a variable, or otherwise used in any context
where any value is legitimate, except that a dispatch to or case on
{\tt void} generates a runtime error.

Variables of the basic classes {\tt Int}, {\tt Bool}, and {\tt String} are 
initialized specially; see Section~\ref{sec-basic}.

\section{Methods}
\label{sec-method}

A method definition has the form
\begin{verbatim}
<id>(<id> : <type>,...,<id> : <type>): <type> is <expr> end;
\end{verbatim}
There may be zero or more formal parameters.  The identifiers used
in the formal parameter list must be distinct.  The type of the
method body must conform to the declared return type.  When a method is
invoked, the formal parameters are bound to the actual arguments
and the expression is evaluated; the resulting value is the meaning
of the method invocation.  A formal parameter
hides any definition of an attribute of the same name.  

To ensure type safety, there are restrictions on the redefinition of
inherited methods.  The rule is simple: If a class {\tt A} inherits a
method {\tt f} from an ancestor class {\tt B}, then {\tt A} may
override the inherited definition of {\tt f} provided the number of
arguments, the types of the formal parameters, and the return type are
exactly the same in both definitions. 

To see why some restriction is necessary on the redefinition of inherited
methods, consider the following example:
\begin{verbatim}
class A is
   f(): Int is 1 end;
end;

class B inherits A is
   f(): String is "1" end;
end;
\end{verbatim}
Let {\tt a} be an object with dynamic type {\tt A}.
Then 
\begin{verbatim}
a.f() + 1
\end{verbatim}
is a well-formed expression with value 2.  However, we cannot
substitute a value of type {\tt B} for {\tt a}, as it would result in
adding a string to a number.  Thus, if methods can be 
redefined arbitrarily, then subclasses may not simply extend the behavior of
their parents, and much of the usefulness of inheritance, as well
as type safety, is lost.

\section{Expressions}
\label{sec-expr}

Expressions are the largest syntactic category in Cool.
\subsection{Constants}

The simplest expressions are constants.  The boolean constants are 
{\tt true} and {\tt false}.  Integer constants are unsigned strings of
digits such as {\tt 0}, {\tt 123}, and {\tt 007}.  String constants 
are sequences of characters enclosed in double quotes, such as
{\tt  "This is a string."}  
String constants may be at most 1024 characters long.  There are
other restrictions on strings; see Section~\ref{lex-struct}.

The constants belong to the basic classes {\tt Bool}, {\tt Int}, and
{\tt String}.
The value of a constant is an object of the appropriate basic class.

\subsection{Identifiers}

The names of local variables, formal parameters of methods,  {\tt self},
and class attributes are all expressions.  The identifier {\tt self} may
be referenced, but it is an error to assign to {\tt self} or to bind
{\tt self} in a {\tt let}, a {\tt case}, or as a formal parameter.
It is also illegal to have attributes named {\tt self}.

Local variables and formal parameters have lexical scope.  Attributes
are visible throughout a class in which they are declared or
inherited, although they may be hidden by local declarations within
expressions.  The binding of an identifier reference is the innermost
scope that contains a declaration for that identifier, or to the
attribute of the same name if there is no other declaration.  The
exception to this rule is the identifier {\tt self}, which is
implicitly bound in every class.

\subsection{Assignment}
An assignment has the form
\begin{verbatim}
<id> <- <expr> 
\end{verbatim}
The static type of the expression must conform to the declared type of
the identifier.  The value is the value of the expression.
The static type of an assignment is the static type of {\tt <expr>}.

\subsection{Dispatch}

There are three forms of dispatch in Cool. 
The three forms differ only in how the called method is selected.
The most commonly used form of dispatch is
\begin{verbatim}
<expr>.<id>(<expr>,...,<expr>)
\end{verbatim}
Consider the dispatch $\tt e_0.f(e_1,\ldots,e_n)$.
To evaluate this expression, the arguments are evaluated in left-to-right
order, from $\tt e_1$ to $\tt e_n$.  Next, $\tt e_0$ is evaluated and
its class {\tt C} noted (if $\tt e_0$ is {\tt void} a runtime error is
generated).  Finally, the method {\tt f} in class {\tt C}
is invoked, with the value of $\tt e_0$ bound to {\tt self} in the body of
{\tt f} and the actual arguments bound to the formals as usual.
The value of the expression is the value returned by the method invocation.

Type checking a dispatch involves several steps.  Assume $\tt e_0$ has
static type {\tt A}.  (Recall that this type is not necessarily the
same as the type {\tt C} above.  {\tt A} is the type inferred by the
type checker; {\tt C} is the class of the object computed at runtime,
which is potentially any subclass of {\tt A}.)  Class {\tt A} must
have a method {\tt f}, the dispatch and the definition of {\tt f} must
have the same number of arguments, and the static type of the $i$th actual
parameter must conform to the declared type of the $i$th formal
parameter.

If {\tt f} has return type {\tt B} and {\tt B} is a class name, then
the static type of the dispatch is {\tt B}.  Otherwise, if {\tt f} has
return type \styp, then the static type of the dispatch is {\tt A}. To see
why this is sound, note that the {\tt self} parameter of the method {\tt f}
conforms to type {\tt A}.  Therefore, because {\tt f} returns \styp,
we can infer that the result must also conform to {\tt A}.  Inferring
accurate static types for dispatch expressions is what justifies including
\styp\ in the Cool type system.


The other forms of dispatch are:
\begin{verbatim}
<id>(<expr>,...,<expr>)
<expr>@<type>.id(<expr>,...,<expr>)
\end{verbatim}
The first form is shorthand for {\tt self.<id>(<expr>,...,<expr>)}.

The second form  provides a way of accessing methods of parent classes 
that have been hidden by redefinitions in child classes.
Instead of using the class of the leftmost expression
to determine the method, the method of the class explicitly specified is
used.  For example, {\tt e@B.f()} invokes the method {\tt f} in
class {\tt B} on the object that is the value of {\tt e}.  For this form
of dispatch, the static type to the left of ``@''must conform to the
type specified to the right of ``@''.

\subsection{Conditionals}
\label{sec-cond}

A conditional has the form
\begin{verbatim}
if <expr> then <expr> else <expr> fi
\end{verbatim}

The semantics of conditionals is standard.  The predicate is evaluated
first.  If the predicate is {\tt true}, then the {\tt then} branch is
evaluated.  If the predicate is {\tt false}, then the {\tt else} branch is
evaluated.  The value of the conditional is the value of the evaluated branch.

The predicate must have static type {\tt Bool}.  The branches may have any static types.
To specify the static type of the conditional, we define an operation $\sqcup$
(pronounced ``join'') on types as follows.  Let {\tt A,B,D} be any types other than \styp.  The {\em least type} of a set of types means the least element
with respect to the conformance relation $\leq$.
\[
\begin{array}{rcl}
\tt A \sqcup B & = & \tt \m{\rm the least type\ } C \m{\rm \ such that\ }
	 A \leq C \m{\rm\ and } B \leq C \\
\tt \styp_D \sqcup \styp_D & = & \tt \styp_D \\
\tt \styp_D \sqcup A & = & \tt D \sqcup A \\
\tt A \sqcup \styp_D & = & \tt A \sqcup D 
\end{array}
\]

Let {\tt T} and {\tt F} be the static types of the branches of the conditional.
Then the static type of the conditional is $\tt T \sqcup F$.

\subsection{Loops}
A loop has the form
\begin{verbatim}
while <expr> loop <expr> pool
\end{verbatim}
The predicate is evaluated before each iteration of the loop.
If the predicate is {\tt false}, the loop terminates and {\tt void}
is returned.  If the predicate is {\tt true},
the body of the loop is evaluated and the process repeats.

The predicate must have static type {\tt Bool}.  The body may have any
static type.  The static type of a loop expression is {\tt Object}.

\subsection{Blocks}
A block has the form
\begin{verbatim}
begin <expr>; ... <expr>; end
\end{verbatim}
The expressions are evaluated in left-to-right order.  Every block has
at least one expression; the value of a block is the value of the last
expression.  The expressions of a block may have any static types.
The static type of a block is the static type of the last expression.

An occasional source of confusion in Cool is the use of semi-colons.
Semi-colons are used as terminators in lists of expressions
(e.g., the block syntax above) and in no other places; see Section~\ref{sec-gram}.

\subsection{Let}
A let expression has the form
\begin{verbatim}
let <id1> : <type1> [ <- <expr1> ], ..., <idn> : <typen> [ <- <exprn> ] in <expr> end
\end{verbatim}

The optional expressions are {\em initialization}; the other
expression is the {\em body}.
A {\tt let} is evaluated as follows.
First {\tt <expr1>} is evaluated and the result bound to {\tt <id1>}.
Then {\tt <expr2>} is evaluated and the result bound to {\tt <id2>},
and so on, until all of the variables in the
{\tt let} are initialized.  (If the initialization of {\tt <idk>} is
omitted, the default initialization of type {\tt <typek>} is used.)
Next the body of the {\tt let} is evaluated.
The value of the {\tt let} is the value of the body.

The {\tt let} identifiers {\tt <id1>,\ldots,<idn>} are visible in the
body of the {\tt let}.  Furthermore, identifiers {\tt <id1>,\ldots,<idk>} are
visible in the initialization of {\tt <idm>} for any {\tt m > k}.

If an identifier is defined multiple times in a {\tt let}, later
bindings hide earlier ones.  Identifiers introduced by {\tt let} also
hide any definitions for the same names in containing scopes.
Every {\tt let} expression must introduce at least one identifier.

The type of an initialization expression must conform to
the declared type of the identifier.  The type of {\tt let} is the type
of the body.

\subsection{Case}
A case expression has the form
\begin{verbatim}
case <expr0> of 
    <id1> : <type1> => <expr1>; 
    . . .
    <idn> : <typen> => <exprn>; 
esac
\end{verbatim}
Case expressions provide runtime type tests on objects. 
First, {\tt expr0} is evaluated and its dynamic type {\tt C} noted
(if {\tt expr0} evaluates to {\tt void} a run-time error is produced).
Next, from among the branches the branch with the least type
{\tt <typek>} such that {\tt C $\leq$ <typek>} is selected. The identifier
{\tt <idk>} is bound to the
value of {\tt <expr0>} and the expression {\tt <exprk>} is evaluated.
The result of the {\tt case} is the value of {\tt <exprk>}.
If no branch can be selected for evaluation, a run-time error is generated.
Every {\tt case} expression must have at least one branch.

%The declared type {\tt <typei>} of
%each branch should conform to the static type of {\tt <expr0>}.
%If the type does not conform, a compiler warning is printed that the branch is
%``dead code'' and can never be executed at runtime.

For each branch, let $\tt T_i$ be the static type of {\tt <expri>}.
The static type of a {\tt case} expression is $\tt \bigsqcup_{1 \leq i \leq
n} T_i$.
The identifier {\tt id} introduced by a branch of a {\tt case} hides any 
variable or attribute definition for {\tt id} visible in the containing scope.

The {\tt case} expression has no special construct for a ``default'' or
``otherwise'' branch.  The same affect is achieved by including a branch
\begin{verbatim}
   x : Object => ...
\end{verbatim}
because every type is $\leq$ to {\tt Object}.

The {\tt case} expression provides programmers a way to insert explicit
runtime type checks in situations where static types inferred by
the type checker are too conservative.  A typical
situation is that a programmer writes an expression $\tt e$ and type
checking infers that $\tt e$ has static type $\tt A$.  However, the programmer
may know that, in fact, the dynamic type of $\tt e$ is always $\tt B$ for
some $\tt B \leq A$.  This information can be captured
using a case expression:
\begin{verbatim}
case e of x : B => ...
\end{verbatim}
In the branch the variable $\tt x$ is bound to the value of $\tt e$ but
has the more specific static type $\tt B$.

\subsection{New}
A {\tt new} expression has the form
\begin{verbatim}
new <type> 
\end{verbatim}
The value is a fresh object of the appropriate class.
If the type is \styp, then the value is a fresh object of the
class of {\tt self} in the current scope.
The static type is {\tt <type>}.


\subsection{Isvoid}
\label{sec-isvoid}

The expression
\begin{verbatim}
isvoid expr
\end{verbatim}
evaluates to {\tt true} if {\tt expr} is {\tt void} and evaluates
to {\tt false} if {\tt expr} is not {\tt void}.


\subsection{Arithmetic and Comparison Operations}

Cool has four binary arithmetic operations: {\tt +, -, *, /}.
The syntax is
\begin{verbatim}
expr1 <op> expr2
\end{verbatim}
To evaluate such an expression first {\tt expr1} is evaluated and
then {\tt expr2}.  The result of the operation is the result of the
expression.

The static types of the two sub-expressions must be {\tt Int}.
The static type of the expression is {\tt Int}.
Cool has only integer division.

Cool has three comparison operations: {\tt <, <=, =}.
For {\tt <} and {\tt <=} the rules are exactly the same as
for the binary arithmetic operations, except that the result
is a {\tt Bool}.  The comparison {\tt =} is a special case.
If either {\tt <expr1>} or {\tt <expr2>} has static type {\tt Int},
{\tt Bool}, or {\tt String}, then the other must have the same static type.
Any other types, including \styp, may be freely compared. 
On non-basic objects, equality simply checks for pointer equality (i.e.,
whether the memory addresses of the objects are the same).  Equality
is defined for {\tt void}.

In principle, there is nothing wrong with permitting equality
tests between, for example, {\tt Bool} and {\tt Int}.  However, such a test
must always be false and almost certainly indicates some sort of
programming error.  The Cool type checking rules catch such errors
at compile-time instead of waiting until runtime.

Finally, there is one arithmetic and one logical unary operator.
The expression {\tt \~{ }<expr>} is the integer complement of {\tt <expr>}.
The expression {\tt <expr>} must have static type {\tt Int} and the entire
expression has static type {\tt Int}.
The expression {\tt not <expr>} is the boolean complement of {\tt <expr>}.
The expression {\tt <expr>} must have static type {\tt Bool} and the entire
expression has static type {\tt Bool}.

\section{Basic Classes}
\label{sec-basic}

\subsection{Object}
The {\tt Object} class is the root of the inheritance graph.
Methods with the following declarations are defined:
\begin{verbatim}
abort() : Object
type_name() : String
copy() : SELF_TYPE
\end{verbatim}
The method {\tt abort} halts program execution with an error message.
The method {\tt type\_name} returns a string with the name of the class
of the object.
The method {\tt copy} produces a {\em shallow} copy of the object.\footnote{A shallow copy of $a$ copies $a$ itself, but does not recursively
copy objects that $a$ points to.}

\subsection{IO}
The {\tt IO} class provides the following methods for performing simple input and output
operations:
\begin{verbatim}
out_string(x : String) : SELF_TYPE
out_int(x : Int) : SELF_TYPE
in_string() : String
in_int() : Int
\end{verbatim}
The methods {\tt out\_string} and {\tt out\_int} print their argument and return
their {\tt self} parameter.  The method {\tt in\_string} reads a string from 
the standard input, up to but not including a newline character.  The method
{\tt in\_int} reads a single integer, which may be preceded by whitespace.
Any characters following the integer, up to and including the next newline,
are discarded by {\tt in\_int}.

A class can make use of the methods in the {\tt IO} class by inheriting
from {\tt IO}.  It is an error to redefine the {\tt IO} class.

\subsection{Int}
\label{sec-int}

The {\tt Int} class provides integers.  There are no methods special
to {\tt Int}.  The default initialization for variables of type {\tt
Int} is 0 (not {\tt void}). It is an error to inherit from or redefine
{\tt Int}.

\subsection{String}
The {\tt String} class provides strings.  The following methods are defined:
\begin{verbatim}
length() : Int
concat(s : String) : String
substr(i : Int, l : Int) : String
\end{verbatim}
The method {\tt length} returns the length of the {\tt self} parameter.
The method {\tt concat} returns the string formed by concatenating
{\tt s} after {\tt self}.  The method {\tt substr} returns the 
substring of its {\tt self} parameter beginning at position {\tt i }
with length {\tt l}; string positions are numbered beginning at 0.
A runtime error is generated if the specified
substring is out of range.  

The default initialization for variables of type {\tt String} is {\tt
""} (not {\tt void}).  It is an error to inherit from or redefine {\tt
String}.

\subsection{Bool}
\label{sec-bool}

The {\tt Bool} class provides {\tt true} and {\tt false}.  
The default initialization for variables of type {\tt Bool} is {\tt false}
(not {\tt void}).
It is an error to inherit from or redefine {\tt Bool}.

\section{Main Class}
\label{sec-main}

Every program must have a class {\tt Main}.
Furthermore, the {\tt Main} class must have a method {\tt main}
that takes no formal parameters.  A program is executed by
evaluating {\tt (new Main).main()}.

The remaining sections of this manual provide a more formal definition
of Cool.  There are four sections covering lexical structure (Section~\ref{lex-struct}), grammar (Section~\ref{sec-gram}), type rules (Section~\ref{sec-typrules}),
and operational semantics (Section~\ref{sec-opsem}).

\section{Lexical Structure}
\label{lex-struct}

The lexical units of Cool are integers, type
identifiers, object identifiers, special notation, strings, keywords,
and white space.

\subsection{Integers, Identifiers, and Special Notation}

Integers are non-empty strings of digits 0-9.  
Identifiers are strings (other than keywords) consisting of letters,
digits, and the underscore character.  Type identifiers begin with a
capital letter; object identifiers begin with a lower case letter.
There are two other identifiers,
{\bf self} and {\bf SELF\_TYPE} that are treated specially by Cool but are not
treated as keywords.  The special syntactic symbols (e.g., parentheses, assignment
operator, etc.) are given in Figure~\ref{fig1}.

\subsection{Strings}

Strings are enclosed in double quotes {\tt "\ldots"}.  Within a string, a
sequence `$\rm \backslash c$' denotes the character `c', with the exception of
the following: \\[.1in]
\begin{tabular}{rl}
$\rm \backslash b$ & backspace \\
$\rm \backslash t$ & tab \\
$\rm \backslash n$ & newline \\
$\rm \backslash f$ & formfeed
\end{tabular} \\[.1in]

A non-escaped newline character may not appear in a string:
\begin{verbatim}
"This is not
OK"
\end{verbatim}
A string may not contain null (character {\tt $\backslash 0$}).
Any other character may be included
in a string.
Strings cannot cross file boundaries.

\subsection{Comments}

There are two forms of comments in Cool.  Any characters between two
dashes `` -- --'' and the next newline (or EOF, if there is no next
newline) are treated as comments.  Comments may also be written by
enclosing text in $(\ast \ldots \ast )$.  The latter form of comment
may be nested.  Comments cannot cross file boundaries.

\subsection{Keywords}

The keywords of cool are:
{\bf begin, class, else, end, false, fi, if, in, inherits, is, isvoid, let, loop, 
pool, then, while, case, esac, new, of, not, true.}  Except for the constants
{\bf true} and {\bf false}, keywords are case insensitive.  To conform to
the rules for other objects, the first letter of {\bf true} and {\bf false}
must be lowercase; the trailing letters may be upper or lower case.

\subsection{White Space}

White space consists of any sequence of the characters: blank, newline, $\tt 
\backslash b, \backslash f, \backslash r, \backslash t, \backslash v$.
\section{Cool Syntax}
\label{sec-gram}


\begin{figure}
\begin{eqnarray*}
\nt{program} & \grammar & \nt{class};^{+} \\
\nt{class} & \grammar & \kw{class} \s \tm{TYPE} \s [ \kw{inherits} \s \tm{TYPE} ] \s \kw{is} \s \nt{feature};^{\ast} \kw{end} \\
\nt{feature} & \grammar & \tm{ID}( \nt{formal},^{\ast} ) : \tm{TYPE} \s \kw{is}
   \s \nt{expr} \s \kw{end} \\
& \gor & \tm{ID} : \tm{TYPE} \s [ \assgn \nt{expr} \s ] \\
\nt{formal} & \grammar & \tm{ID} : \tm{TYPE} \\
\nt{expr} & \grammar & \tm{ID} \assgn \nt{expr} \\
 & \gor & \nt{expr}[@\tm{TYPE}].\tm{ID}(\nt{expr},^{\ast}) \\
 & \gor & \tm{ID}(\nt{expr},^{\ast}) \\
& \gor & \kw{if} \s \nt{expr} \s \kw{then} \s \nt{expr} \s \kw{else} \s \nt{expr} \s \kw{fi}  \\
& \gor & \kw{while} \s \nt{expr} \s \kw{loop} \s \nt{expr} \s \kw{pool} \\
& \gor & \kw{begin} \s \nt{expr};^+ \kw{end} \\
& \gor & \kw{let} \s \{ \tm{ID} : \tm{TYPE} \s [ \assgn \nt{expr} \s ],\}^+ \s \kw{in} \s
	\nt{expr} \s  \kw{end} \\
& \gor & \kw{case} \s \nt{expr} \s \kw{of} \s 
		\{ \tm{ID} : \tm{TYPE} => \nt{expr}; \}^+
	 \kw{esac} \\
& \gor & \kw{new} \s \tm{TYPE} \\
& \gor & \kw{isvoid} \s \nt{expr} \\
& \gor & \nt{expr} \,+\, \nt{expr} \\
& \gor & \nt{expr} \,-\, \nt{expr} \\
& \gor & \nt{expr} \,\ast\, \nt{expr} \\
& \gor & \nt{expr} \,/\, \nt{expr} \\
& \gor & \m{\~{ }}\nt{expr} \\
& \gor & \nt{expr} < \nt{expr} \\
& \gor & \nt{expr} <= \nt{expr} \\
& \gor & \nt{expr} = \nt{expr} \\
& \gor & \kw{not} \s \nt{expr} \\
& \gor & (\nt{expr}) \\
& \gor & \tm{ID} \\
& \gor & \m{integer} \\
& \gor & \m{string} \\
& \gor & \kw{true} \\
& \gor & \kw{false}
\end{eqnarray*}
\caption{Cool syntax.}
\label{fig1}
\end{figure}

Figure~\ref{fig1} provides a specification of Cool syntax.  The
specification is not context-free; for convenience, we also use some
regular expression notation.  Specifically, $A^{\ast}$ means zero or
more $A$'s in succession; $A^+$ means one or more $A$'s.  The special
notation $A;^{\ast}$ means zero or more $A$'s terminated by
semicolons.  The special notation $A,^{\ast}$ means zero or more $A$'s
separated by commas.  Separators differ from terminators in that the
terminator always appears at the end of the list; a separator is never
included at the end of a list.  In Cool, semicolons are always
terminators and commas are always separators.  Items in square brackets
$[\ldots]$ are optional.  Curly braces are not part of Cool; they are
used in the grammar as a meta-symbol to show association of grammar
operations.

\subsection{Precedence}
The precedence of infix binary and prefix unary operations, 
from highest to lowest, is given by
the following table:
\begin{verbatim}
.
@
~
isvoid
* /
+ -
<=  <  =
not
<-
\end{verbatim}
All binary operations are left-associative, with the exception of 
assignment, which is right-associative, and the three comparison operations,
which do not associate.

\section{Type Rules}
\label{sec-typrules}

This section formally defines the type rules of Cool.
The type rules define the type of every Cool expression 
in a given context.  The context is the {\em type environment},
which describes the type of every unbound identifier appearing in an
expression.  The type environment is described in Section~\ref{sec-typenv}.
Section~\ref{sec-typr} gives the type rules.

\subsection{Type Environments}
\label{sec-typenv}

To a first approximation, type checking in Cool can be thought of as
a bottom-up algorithm: the type of an expression $e$ is computed from the
(previously computed) types of $e$'s subexpressions.  For example,
an integer {\tt 1} has type {\tt Int}; there are no subexpressions in
this case.  As another example, if $\tt e_n$ has type $\tt X$, then
the expression $\tt \m{\tt begin}\ e_1; \ldots; e_n\ \m{\tt end}$ has 
type $\tt X$.

A complication arises in the case of an expression $\tt v$, where $\tt
v$ is an object identifier.  It is not possible to say what the type of $\tt
v$ is in a strictly bottom-up algorithm; we need to know the type
declared for $\tt v$ in the larger expression.  Such a declaration
must exist for every object identifier in valid Cool programs.

To capture information about the types of identifiers, we use a
{\em type environment}.  The environment consists of three parts:
a method environment $M$, an object environment $O$, and 
the name of the current class in which the expression appears.  
The method environment and object environment are both functions (also
called {\em mappings}).  The object environment is a function of the
form
\[  O(v) = T \]
which assigns the type $ T$ to object identifier $ v$.
The method environment is more complex; it is a function of the form
\[  M(C,f) = (T_1,\ldots,T_{n-1},T_n) \]
where $ C$ is a class name (a type), $f$ is a method name,
and $ t_1,\ldots,t_n$ are types.  The tuple of types is
the {\em signature} of the method.  The interpretation of signatures is
that in class $ C$ the method $ f$ has formal parameters of
types $ (t_1,\ldots,t_{n-1})$---in that order---and a return type $t_n$.


Two mappings are required instead of one because object names and method
names do not clash---i.e., there may be a method and an object identifier
of the same name.

The third component of the type environment is the name of the current
class, which is needed for type rules involving {\tt SELF\_TYPE}.

Every expression $e$ is type checked in a type environment;
the subexpressions of $e$ may be type checked in the same environment or,
if $e$ introduces a new object identifier, in a modified environment.
For example, consider the expression
\begin{center}
\protect\begin{verbatim}
  let c : Int <- 33 in
    ...
  end
\end{verbatim}
\end{center}
The \C{let} expression introduces a new variable {\tt c} with type {\tt Int}.
Let $O$ be the object component of the type environment for the \C{let}.
Then the body of the \C{let} is type checked in the object type environment
\[ O[Int/c] \]
where the notation $O[T/c]$ is defined as follows:
\begin{eqnarray*}
O[T/c](c) & = & T \\
O[T/c](d) & = & O(d)\ \ \mif d \neq c
\end{eqnarray*}

\subsection{Type Checking Rules}
\label{sec-typr}

The general form a type checking rule is:
$$
\tcrule{\vdots}{O,M,C \vdash e : T}{}
$$ 
The rule should be read: In the type environment for objects $O$,
methods $M$, and containing class $C$, the expression $e$ has type
$T$.  The dots above the horizontal bar stand for other statements
about the types of sub-expressions of $e$.  These other statements are
hypotheses of the rule; if the hypotheses are satisfied, then the
statement below the bar is true.

The rule for object identifiers is simply that if the environment
assigns an identifier $ Id $ type $T$, then $ Id $ has type $T$.
$$
\tcrule{O(Id) = T}{O,M,C \vdash Id : T}{[Var]}
$$

The rule for assignment to a variable is more complex:
$$
\tcrule{\begin{array}{l}
	O(Id) = T \\
	O,M,C \vdash e_1 : T' \\
	T' \leq T
	\end{array}}
       {O,M,C \vdash Id \leftarrow e_1 : T'}
       {[ASSIGN]}
$$
Note that this type rule---as well as others---use the conformance relation
$\leq$ (see Section~\ref{sec-inherit}).
The rule says that the assigned expression $e_1$ must have a type $T'$ that
conforms to the type $T$ of the identifier $ Id $ in the type environment.
The type of the whole expression is $T'$.

The type rules for constants are all easy:
$$
\tcrule{}
       {O,M,C \vdash true : Bool}
       {[True]}
$$

$$
\tcrule{}
       {O,M,C \vdash false : Bool}
       {[False]}
$$

$$
\tcrule{i \m{\ is an integer constant}}
       {O,M,C \vdash i : Int}
       {[Int]}
$$

$$
\tcrule{s \m{\ is a string constant}}
       {O,M,C \vdash s : String}
       {[String]}
$$

There are two cases for \C{new}, one for {\tt new SELF\_TYPE} and
one for any other form:
$$
\tcrule{
T' = \left\{
	\begin{array}{rl}
		{\tt SELF\_TYPE}_C & \mif T = {\tt SELF\_TYPE} \\
		T & \ow
	\end{array}
     \right.}
{O,M,C \vdash new\ T: T'}
{[New]}
$$

Dispatch expressions are the most complex to type check.
$$
\tcrule{
\begin{array}{l}
O,M,C \vdash e_0 : T_0 \\
O,M,C \vdash e_1 : T_1 \\
\vdots \\
O,M,C \vdash e_n : T_n \\
T_0' = \left\{ 
	\begin{array}{rl}
		C & \m{\ if\ } T_0 = \m{\tt SELF\_TYPE}_C \\
		T_0 & \m{otherwise}
	\end{array}
	\right. \\
M(T_0',f) = (T_1',\ldots,T_n',T_{n+1}') \\
T_i \leq T_i'\ \ \ 1 \leq i \leq n \\
T_{n+1} = \left\{ 
	\begin{array}{rl}
		T_0 & \m{\ if\ } T_{n+1}' = \m{\tt SELF\_TYPE} \\
		T_{n+1}' & \m{otherwise}
	\end{array}
	\right.
\end{array}
}
{O,M,C \vdash e_0.f(e_1,\ldots,e_n) : T_{n+1}}
{[Dispatch]}
$$

$$
\tcrule{
\begin{array}{l}
O,M,C \vdash e_0 : T_0 \\
O,M,C \vdash e_1 : T_1 \\
\vdots \\
O,M,C \vdash e_n : T_n \\
T_0 \leq T \\
M(T,f) = (T_1',\ldots,T_n',T_{n+1}') \\
T_i \leq T_i'\ \ \ 1 \leq i \leq n \\
T_{n+1} = \left\{ 
	\begin{array}{rl}
		T_0 & \m{\ if\ } T_{n+1}' = \m{\tt SELF\_TYPE} \\
		T_{n+1}' & \m{otherwise}
	\end{array}
	\right.
\end{array}
}
{O,M,C \vdash e_0@T.f(e_1,\ldots,e_n) : T_{n+1}}
{[StaticDispatch]}
$$


To type check a dispatch, each of the subexpressions must
first be type checked.  The type $T_0$ of $e_0$ determines the
which declaration of the method $f$ is used.  The argument types of the
dispatch must conform to the declared argument types.  Note that the
type of the result of the dispatch is either the declared return type
or $T_0$ in the case that the declared return type is $\tt SELF\_TYPE$.
The only difference in type checking a static dispatch is that the
class $T$ of the method $f$ is given in the dispatch, and the type
$T_0$ must conform to $T$.

The type checking rules for \C{if} and \C{begin}-\C{end}
expressions are straightforward.  See Section~\ref{sec-cond} for
the definition of the $\sqcup$ operation. 
$$
\tcrule{
\begin{array}{l}
O,M,C \vdash e_1 : Bool \\
O,M,C \vdash e_2 : T_2 \\
O,M,C \vdash e_3 : T_3
\end{array}}
{O,M,C \vdash \m{if } e_1 \m{ then } e_2 \m{ else } e_3 \m{ fi} : T_2 \sqcup T_3}
{[If]}
$$

$$
\tcrule{
\begin{array}{l}
O,M,C \vdash e_1 : T_1 \\
O,M,C \vdash e_2 : T_2 \\
\vdots \\
O,M,C \vdash e_n : T_n 
\end{array}}
{O,M,C \vdash \m{begin } e_1; e_2; \ldots e_n; \m{ end} : T_n}
{[Sequence]}
$$

The \C{let} rule has some interesting aspects.
$$
\tcrule{
\begin{array}{l}
T_0' = \left\{
	\begin{array}{rl}
		{\tt SELF\_TYPE}_C & \mif\ T_0 = {\tt SELF\_TYPE} \\
		T_0 & \ow
	\end{array}
	\right. \\
O,M,C \vdash e_1 : T_1 \\
T_1 \leq T_0' \\
O[T_0'/x],M,C \vdash e_2 : T_2 
\end{array}}
{O,M,C \vdash \m{let } x : T_0 \leftarrow e_1 \m{ in } e_2 \m{ end}: T_2}
{[Let-Init]}
$$
First, the initialization $e_1$ is type checked in an environment
without a definition for $x$.  Thus, the variable $x$ cannot be used
in $e_1$ unless it already has a definition in an outer scope.
Second, the body $e_2$ is type checked in the environment $O$ extended
with the typing $x:T_0'$.
Third, note that the type of $x$ may be {\tt SELF\_TYPE}.

$$
\tcrule{
\begin{array}{l}
T_0' = \left\{
	\begin{array}{rl}
		{\tt SELF\_TYPE}_C & \mif\ T_0 = {\tt SELF\_TYPE} \\
		T_0 & \ow
	\end{array}
	\right. \\
O[T_0'/x],M,C \vdash e_1 : T_1 
\end{array}}
{O,M,C \vdash \m{let } x : T_0 \m{ in } e_1 \m{ end} : T_1}
{[Let-No-Init]}
$$
The rule for \C{let} with no initialization simply omits the conformance
requirement.  We give
type rules only for a \C{let} with a single variable.
Typing a multiple \C{let}
\[ \rm let\ x_1 : T_1\ [\leftarrow e_1],\ x_2: T_2\ [\leftarrow e2], \ldots,\ x_n :T_n\ [\leftarrow e_n]\ in\ e\ end \]
is defined to be the same as typing
\[
\rm let\ x_1 : T_1\ [\leftarrow e_1]\ in\ (let\ x_2 :T_2\ [\leftarrow e_2],\ldots,\ x_n : T_n\ [\leftarrow e_n]\ in\ e\ end)\ end
\]

$$
\tcrule{
\begin{array}{l}
O,M,C \vdash e_0 : T_0 \\
O[T_1/x_1], M, C \vdash e_1 : T_1' \\
\vdots \\
O[T_n/x_n], M, C \vdash e_n : T_n'
\end{array}}
{O,M,C \vdash \m{case } e_0 \m{ of } x_1:T_1 \Rightarrow e_1; \ldots
				     x_n:T_n \Rightarrow e_n; \m{ esac} :
	\bigsqcup_{1 \leq i \leq n} T_i'}
{[Case]}
$$

Each branch of a \C{case} is type checked in an environment where
variable $x_i$ has type $T_i$.  The type of the entire \C{case} is
the join of the types of its branches.  The variables declared on 
each branch of a \C{case} must all have distinct types.

$$
\tcrule{
\begin{array}{l}
O,M,C \vdash e_1 : Bool \\
O,M,C \vdash e_2: T_2 
\end{array}}
{O,M,C \vdash \m{while } e_1 \m{ loop } e_2 \m{ pool} : Object}
{[Loop]}
$$
The predicate of a loop must have type $Bool$; the type of the entire
loop is always $Object$. An \C{isvoid} test has type $Bool$:
$$
\tcrule{O,M,C \vdash e_1 : T_1}
{O,M,C \vdash \m{isvoid } e_1 : Bool}
{[Isvoid]}
$$

With the exception of the rule for equality,
the type checking rules for the primitive
logical, comparison, and arithmetic operations are easy.
$$
\tcrule{
O,M,C \vdash e_1 : Bool}
{O,M,C \vdash \neg e_1 : Bool}
{[Not]}
$$

$$
\tcrule{
\begin{array}{l}
O,M,C \vdash e_1 : Int \\
O,M,C \vdash e_2 : Int \\
op \in \{ <, \leq \}
\end{array}}
{O,M,C \vdash e_1\ op\  e_2 : Bool}
{[Compare]}
$$

$$
\tcrule{
O,M,C \vdash e_1 : Int}
{O,M,C \vdash \mbox{\~{}} e_1 : Int}
{[Neg]}
$$

$$
\tcrule{
\begin{array}{l}
O,M,C \vdash e_1 : Int \\
O,M,C \vdash e_2 : Int \\
op \in \{ \ast,+,-,/ \}
\end{array}}
{O,M,C \vdash e_1\ op\  e_2 : Int}
{[Arith]}
$$

The wrinkle in the rule for equality is that any types may be
freely compared except {\tt Int, String} and {\tt Bool}, which
may only be compared with objects of the same type.
$$
\tcrule{
\begin{array}{l}
O,M,C \vdash e_1 : T_1 \\
O,M,C \vdash e_2 : T_2 \\ 
T_1 \in \{ Int, String, Bool \} \Or T_2 \in \{ Int, String, Bool \} \Rightarrow T_1 = T_2
\end{array}}
{O,M,C \vdash e_1 = e_2 : Bool}
{[Equal]}
$$

The final cases are type checking rules for attributes and methods.
For a class $C$, let the object environment $O_C$ give the types of all
attributes of $C$ (including any inherited attributes).  More formally,
if $x$ is an attribute (inherited or not) of $C$, and the declaration of
$x$ is $x:T$, then
\[ O_C(x) = \left\{
	\begin{array}{rl}
		{\tt SELF\_TYPE}_C & \mif\ T = {\tt SELF\_TYPE} \\
		T & \ow
	\end{array}
	\right.
\]

The method environment $M$ is global to the entire program and defines
for every class $C$ the signatures of all of the methods of $C$
(including any inherited methods).

The two rules for type checking attribute defininitions are similar
the rules for \C{let}.  The essential difference is that
attributes are visible within their initialization expressions.  
Note that {\tt self} is bound in the initialization.
$$
\tcrule{
\begin{array}{l}
O_C(x) = T_0 \\
O_C[{\tt SELF\_TYPE}_C/\m{\em self}],M,C \vdash e_1 : T_1 \\
T_1 \leq T_0 \\
\end{array}}
{O_C,M,C \vdash x : T_0 \leftarrow e_1;}
{[Attr-Init]}
$$

$$
\tcrule{O_C(x) = T}
{O_C,M,C \vdash x : T;}
{[Attr-No-Init]}
$$

The rule for typing methods checks the body of the method in an
environment where $O_C$ is extended with bindings for the formal
parameters and {\tt self}. The type of the method body must conform to
the declared return type.

$$
\tcrule{
\begin{array}{l}
M(C,f) = (T_1,\ldots,T_n,T_0) \\
O_C[{\tt SELF\_TYPE}_C/\m{\em self}][T_1/x_1]\ldots[T_n/x_n], M, C \vdash e : T_0' \\
T_0' \leq \left\{
		\begin{array}{rl}
			{\tt SELF\_TYPE}_C & \mif T_0 = {\tt SELF\_TYPE} \\
			T_0 & \ow
		\end{array}
	  \right.
\end{array}}
{O_C,M,C \vdash f(x_1:T_1,\ldots,x_n:T_n) : T_0 \m{ is } e \m{ end; }}
{[Method]}
$$

\section{Operational Semantics}
\label{sec-opsem}

This section contains a mostly formal presentation of the operational
semantics for the Cool language. The operational semantics define for
every Cool expression what value it should produce in a given context.
The context has three components: an environment, a store, and a self
object. These components are described in the next section. Section
\ref{objectsyntax} defines the syntax used to refer to Cool
objects, and Section \ref{classdefs} defines the syntax used to refer to
class definitions.

Keep in mind that a formal semantics is a specification only---it does
not describe an implementation.  The purpose of presenting the formal
semantics is to make clear all the details of the behavior of Cool 
expressions.  How this behavior is implemented is another matter.

\subsection{Environment and the Store}

Before we can present a semantics for Cool we need a number
of concepts and a considerable amount of notation.
An {\em environment} is a mapping of variable identifiers to
{\em locations}.  Intuitively, an environment tells us for a given
identifier the address of the memory location where that identifier's
value is stored.  For a given expression, the environment must
assign a location to all identifiers to which the expression may refer.
 For the expression, e.g., $a + b$, we need an environment that maps
$a$ to some location and $b$ to some location. We'll use the following syntax
to describe environments, which is very similar to the syntax of type
assumptions used in Section~\ref{sec-typrules}.
\[
	E = [ a:l_1, b:l_2]
\]
This environment maps $a$ to location $l_1$, and $b$ to location
$l_2$.

The second component of the context for the evaluation of an
expression is the {\em store}. The store maps locations to values,
where values in Cool are just objects.  Intuitively, a store
tells us what value is stored in a given memory location.
For the moment, assume all values are integers.
A store is similar to an environment:
\[
	S = [ l_1\rightarrow 55, l_2\rightarrow 77 ]
\]
This store maps location $l_1$ to value $55$ and location $l_2$
to value $77$.

Given an environment and a store, the value of an
identifier can be found by first looking up the location that the identifier maps
to in the environment and then looking up the location in the store. 
\[
\begin{array}{rcl}
	E(a) &=& l_1 \\
	S(l_1) & = & 55
\end{array}
\]

Together, the environment and the store define the execution state at
a particular step of the evaluation of a Cool expression. The double
indirection from identifiers to locations to values allows us to model
variables. Consider what happens if the value $99$ is assigned
variable $a$ in the environment and store defined above. Assigning to a
variable means changing the value to which it refers but not its
location. To perform the assignment, we look up the location for
$a$ in the environment $E$ and then change the mapping for the
obtained location to the new value, giving a new store $S'$.

\[
\begin{array}{rcl}
	E(a) & = & l_1 \\
	S'   & = & S[99/l_1]
\end{array}
\]
The syntax $S[v/l]$ denotes a new store that is identical to
the store $S$, except that $S'$ maps location $l$ to value
$v$. For all locations $l'$ where $l'\not=l$, we still have
$S'(l')=S(l')$.


The store models the contents of memory of the computer during program execution.
Assigning to a variable modifies the store.

There are also situations in which the environment is modified.
Consider the following Cool fragment:

\begin{center}
\begin{verbatim}
  let c : Int <- 33 in
    c
  end
\end{verbatim}
\end{center}

When evaluating this expression, we must introduce the new
identifier $c$ into the environment before evaluating the body of
the \C{let}. If the current environment and state are $E$ and
$S$, then we create a new environment $E'$ and a new store $S'$
defined by:
\[
\begin{array}{rcl}
	l_c & = & newloc(S)\\
	E' & = & E[l_c/c]\\
	S' & = & S[33/l_c]
\end{array}
\]
The first step is to allocate a location for the variable $c$.  The
location should be fresh, meaning that the current store does not have
a mapping for it. The function $newloc()$ applied to a store gives us
an unused location in that store. We then create a new environment
$E'$, which maps $c$ to $l_c$ but also contains all of the mappings of
$E$ for identifiers other than $c$. Note that if $c$ already has a
mapping in $E$, the new environment $E'$ hides this old mapping.  We
must also update the store to map the new location to a value. In this
case $l_c$ maps to the value $33$, which is the initial value for $c$
as defined by the let-expression.  

The example in this subsection oversimplifies Cool environments and
stores a bit, because simple integers are not Cool values.
Even integers are full-fledged objects in Cool.


\subsection{Syntax for Cool Objects}
\label{objectsyntax}
Every Cool value is an object.
 Objects contain a list of
named attributes, a bit like records in C.  In addition, each object belongs to a class. We use
the following syntax for values in Cool:
\[
	v = X(a_1=l_1,a_2=l_2,\ldots,a_n=l_n)
\]
Read the syntax as follows: The value $v$ is a member of class $X$ containing the attributes
$a_1, \ldots, a_n$ whose locations are $l_1, \ldots, l_n$. Note that
the attributes have an associated location. Intuitively this means
that there is some space in memory reserved for each attribute.

For base objects of Cool (i.e., \C{Int}s, \C{String}s, and \C{Bool}s) we use a
special case of the above syntax. Base objects have a class
name, but their attributes are not like attributes of normal
classes, because they cannot be modified. Therefore, we describe
base objects using the following syntax:

\[
\begin{array}{l}
	Int(5)\\
	Bool(true)\\
	String(4,\dq Cool \dq)
\end{array}
\]

For \C{Int}s and \C{Bool}s, the meaning is obvious. \C{String}s contain two
parts, the length and the actual sequence of ASCII characters.

\subsection{Class definitions}
\label{classdefs}
In the rules presented in the next section, we need a way to refer to
the definitions of attributes and methods for classes. 
Suppose we have the following Cool class
definition:
\begin{verbatim}
   class B is
      s : String <- "Hello";
      g (y:String) : Int is
         y.concat(s)
      end;
      f (x:Int) : Int is
         x+1
      end;
   end;

   class A inherits B is
      a : Int;
      b : B <- new B;
      f(x:Int) : Int is
         x+a
      end;
   end;
\end{verbatim}

Two mappings, called {\em class} and {\em implementation}, are associated 
with class definitions.  The {\em class} mapping is
used to get the attributes, as well as their types and initializations, of a
particular class:
\[
	class(A) = (s : String \leftarrow \dq Hello \dq,\ a : Int
\leftarrow 0,\ b : B \leftarrow new\ B)
\]
Note that the information for class $A$ contains everything that it
inherited from class $B$, as well as its own definitions. If $B$ had
inherited other attributes, those attributes would also appear in the information
for $A$. The attributes are listed in the order they are inherited
and then in source order:
all the attributes from the greatest ancestor are listed first in the
order in which they textually appear, then the attributes of the next greatest
ancestor, and so on, on down to the attributes defined in the particular class.
We rely on this order in describing how new objects are initialized.

The general form of a class mapping is:
\[
	\classmap{X}{\attr{a_1}{T_1}{e_1},\ \ldots,\ \attr{a_n}{T_n}{e_n}}
\]
Note that every attribute has an initializing expression, even if the
Cool program does not specify one for each attribute. The default
initialization for a variable or attribute is the {\em default} of
its type.  The default of {\tt Int} is 0, the default of {\tt String} 
is $\dq\dq$, the default of {\tt Bool} is {\tt false}, and the default
of any other type is {\tt void}.\footnote{A tiny point: We are
allowing {\tt void} to be used as an expression here.  There is no
expression for {\tt void} available to Cool programmers.}
The default of type $T$ is 
written $D_T$.

The implementation mapping gives information about the methods of
a class. For the above example, {\em implementation} of A  is defined as
follows:

\[
\begin{array}{rcl}
	implementation(A,f) & = & (x, x+a) \\
	implementation(A,g) & = & (y, y.concat(s))
\end{array}
\]

In general, for a class $X$ and a method $m$,
\[
	\mbox{\em implementation}(X,m) = (x_1, x_2, \ldots, x_n, e_{body})
\]
specifies that method $m$ when invoked from class $X$, has formal
parameters $x_1, \ldots, x_n$, and the body of the method is expression
$e_{body}$.

\subsection{Operational Rules}

  Equipped with environments, stores, objects, and class
definitions, we can now attack the operational semantics for Cool.
The operational semantics is described by rules similar to the rules
used in type checking. The general form of the rules is:
$$
\osrule{so} S E {e_1} v {S'} \vdots{}
$$
The rule should be read as: In the context where {\em self} is the object
$so$, the store is $S$, and the environment is $E$,
 the expression $e_1$ evaluates to
object $v$ and the new store is $S'$.
The dots above the horizontal bar stand for other statements about the
evaluation of sub-expressions of $e_1$.

Besides an environment and a store, the evaluation context
contains a self object $so$. The self object is just the object to which
the identifier {\tt self} refers if {\tt self} appears in the
expression. We do not place {\tt self} in the environment and store because
{\tt self} is not a variable---it cannot be assigned to.
Note that the rules specify a new store after the evaluation of an
expression. The new store contains all changes to memory resulting
as side effects of evaluating expression $e_1$.

The rest of this section presents and briefly discusses each of the operational
rules.  A few cases are not covered; these are discussed at the end of the section.
$$
\osrule{so}{S_1} E {Id\leftarrow e_1}{v_1}{S_3}
	{\begin{array}{l}
	 \ossimple{so}{S_1} E {e_1}{v_1}{S_2}\\
	 E(Id) = l_1\\
	 S_3 = S_2[v_1/l_1]
	 \end{array}}{[Assign]}
$$
An assignment first evaluates the expression on the right-hand side, yielding
a value $v_1$.  This value is stored in memory at the address for the
identifier.

The rules for identifier references, {\tt self}, and constants are straightforward:
$$
\osrule{so}{S} E {Id}{v}{S}
	{\begin{array}{l}
	 E(Id) = l\\
	 S(l) = v
	 \end{array}}{[Var]}
$$

$$
\osrule{so}{S} E {\mbox{self}}{so}{S}{}{[Self]}
$$

$$
\osrule{so}{S} E {\mbox{true}}{Bool(true)}{S}{}{[True]}
$$

$$
\osrule{so}{S} E {\mbox{false}}{Bool(false)}{S}{}{[False]}
$$

$$
\osrule{so}{S} E {i}{Int(i)}{S}{i\mbox{ is an integer constant}}{[Int]}
$$

$$
\osrule{so}{S} E {s}{String(l,s)}{S}
	{\begin{array}{l}
	 s\mbox{ is a string constant}\\
	l = length(s)
	 \end{array}}{[String]}
$$

A {\tt new} expression is more complicated than one might expect:
$$
\osrule{so}{S_1} E {\mbox{new }T}{v_1}{S_3}
	{\begin{array}{l}
	T_0 = \left\{ \begin{array}{rl}
			  X & \mif T = \m{\tt SELF\_TYPE} \m{ and }
					v_1 = X(\ldots) \\
			  T & \ow
			\end{array}
		\right. \\
	\classmap{T_0}{\attr{a_1}{T_1}{e_1},\dots,\attr{a_n}{T_n}{e_n}}\\
	l_i = newloc(S_1),\mbox{ for $i=1\ldots n$}\\
	v_1 = T_0(a_1=l_1,\ldots,a_n=l_n)\\
	S_2 = S_1[D_{T_1}/l_1,\ldots,D_{T_n}/l_n]\\
	\ossimple{v_1}{S_2}{[a_1:l_1,\ldots,a_n:l_n]}{\mbox{begin
	}a_1\leftarrow e_1; \ldots; a_n\leftarrow e_n;\mbox{ end}}{v_2}{S_3}
	 \end{array}}{[New]}
$$
The tricky thing in a {\tt new} expression is to initialize the attributes
in the right order.  Note also that, during initialization, attributes are
bound to the default of the appropriate class.

$$
\osrule{so}{S_1} E {e_0.f(e_1,\ldots,e_n)}{v_{n+1}}{S_{n+4}}
	{\begin{array}{l}
	\ossimple{so}{S_1}{E}{e_1}{v_1}{S_2}\\
	\ossimple{so}{S_2}{E}{e_2}{v_2}{S_3}\\
	\vdots{}\\
	\ossimple{so}{S_{n}}{E}{e_n}{v_n}{S_{n+1}}\\
	\ossimple{so}{S_{n+1}}{E}{e_0}{v_0}{S_{n+2}}\\
	v_0 = X(a_1=l_{a_1},\ldots,a_m=l_{a_m})\\
	implementation(X,f) = (x_1,\ldots,x_n,e_{n+1})\\
	l_{x_i} = newloc(S_{n+2}), \mbox{ for $i=1\ldots n$}\\
	S_{n+3} = S_{n+2}[v_1/l_{x_1},\ldots,v_n/l_{x_n}]\\
	\ossimple{v_0}{S_{n+3}}{[a_1:l_{a_1},\ldots,a_m:l_{a_m},\,
	x_1:l_{x_1},\ldots,x_n:l_{x_n}]}{e_{n+1}}{v_{n+1}}{S_{n+4}} 
	 \end{array}}{[Dispatch]}
$$


$$
\osrule{so}{S_1} E {e_0@T.f(e_1,\ldots,e_n)}{v_{n+1}}{S_{n+4}}
	{\begin{array}{l}
	\ossimple{so}{S_1}{E}{e_1}{v_1}{S_2}\\
	\ossimple{so}{S_2}{E}{e_2}{v_2}{S_3}\\
	\vdots{}\\
	\ossimple{so}{S_{n}}{E}{e_n}{v_n}{S_{n+1}}\\
	\ossimple{so}{S_{n+1}}{E}{e_0}{v_0}{S_{n+2}}\\
	v_0 = X(a_1=l_{a_1},\ldots,a_m=l_{a_m})\\
	implementation(T,f) = (x_1,\ldots,x_n,e_{n+1})\\
	l_{x_i} = newloc(S_{n+2}), \mbox{ for $i=1\ldots n$}\\
	S_{n+3} = S_{n+2}[v_1/l_{x_1},\ldots,v_n/l_{x_n}]\\
	\ossimple{v_0}{S_{n+3}}{[a_1:l_{a_1},\ldots,a_m:l_{a_m},\,
	x_1:l_{x_1},\ldots,x_n:l_{x_n}]}{e_{n+1}}{v_{n+1}}{S_{n+4}} 
	 \end{array}}{[StaticDispatch]}
$$

The two dispatch rules do what one would expect.  The arguments are
evaluated and saved.  Next, the expression on the left-hand side of
the ``.'' is evaluated.  In a normal dispatch, the class of this expression
is used to determine the method to invoke; otherwise the class is specified
in the dispatch itself.

$$
\osrule{so}{S_1} E {\mbox{if }e_1\mbox{ then }e_2\mbox{ else }e_3\m{ fi}}{v_2}{S_3}
	{\begin{array}{l}
	\ossimple{so}{S_1}{E}{e_1}{Bool(true)}{S_2}\\
	\ossimple{so}{S_2}{E}{e_2}{v_2}{S_3}\\
	 \end{array}}{[If-True]}
$$

$$
\osrule{so}{S_1} E {\mbox{if }e_1\mbox{ then }e_2\mbox{ else }e_3\m{ fi}}{v_3}{S_3}
	{\begin{array}{l}
	\ossimple{so}{S_1}{E}{e_1}{Bool(false)}{S_2}\\
	\ossimple{so}{S_2}{E}{e_3}{v_3}{S_3}\\
	 \end{array}}{[If-False]}
$$
There are no surprises in the if-then-else rules.  Note that value of
the predicate is a \C{Bool} object, not a boolean.

$$
\osrule{so}{S_1} E {\mbox{begin	}e_1; e_2;\ldots; e_n;\mbox{ end}}{v_n}{S_{n+1}}
	{\begin{array}{l}
	\ossimple{so}{S_1}{E}{e_1}{v_1}{S_2}\\
	\ossimple{so}{S_2}{E}{e_2}{v_2}{S_3}\\
	\vdots{}\\
	\ossimple{so}{S_{n}}{E}{e_n}{v_n}{S_{n+1}}\\
	 \end{array}}{[Sequence]}
$$
Blocks are evaluated from the first expression to the last expression, in
order.  The result is the result of the last expression.

$$
\osrule{so}{S_1} E {\mbox{let }Id : T_1\leftarrow e_1\mbox{ in } e_2\mbox{ end}}{v_2}{S_{4}}
	{\begin{array}{l}
	\ossimple{so}{S_1}{E}{e_1}{v_1}{S_2}\\
	l_1 = newloc(S_2)\\
	S_3 = S_2[v_1/l_1]\\
	E' = E[l_1/Id]\\
	\ossimple{so}{S_3}{E'}{e_2}{v_2}{S_4}\\
	 \end{array}}{[Let]}
$$
A \C{let} evaluates any initialization code, assigns the result to the variable
at a fresh location, and evaluates the body of the \C{let}.  (If there
is no initialization, the variable is initialized to the default value
of $T_1$.) We give the
operational semantics only for the case of \C{let} with a single variable.
The semantics of a multiple \C{let}
\[ \rm let\ x_1 : T_1 \leftarrow e_1,\ x_2: T_2 \leftarrow e2, \ldots,\ x_n :T_n \leftarrow e_n\ in\ e\ end \]
is defined to be the same as
\[
\rm let\ x_1 : T_1 \leftarrow e_1\ in\ (let\ x_2 :T_2 \leftarrow e_2,\ldots,\ x_n : T_n \leftarrow e_n\ in\ e\ end)\ end
\]



$$
\osrule{so}{S_1} E {\mbox{case }e_0\mbox{ of }Id_1:T_1\Rightarrow e_1;\ldots;Id_n:T_n\Rightarrow e_n;\mbox{ esac}}{v_1}{S_{4}}
	{\begin{array}{l}
	\ossimple{so}{S_1}{E}{e_0}{v_0}{S_2}\\
	v_0 = X(\ldots)\\
	T_i = \mbox{closest ancestor of $X$ in $\{T_1,\ldots,T_n\}$}\\
	l_0 = newloc(S_2)\\
	S_3 = S_2[v_0/l_0]\\
	E' = E[l_0/Id_i]\\
	\ossimple{so}{S_3}{E'}{e_i}{v_1}{S_4}\\
	 \end{array}}{[Case]}
$$
Note that the \C{case} rule requires that the class hierarchy be available in
some form at runtime, so that the correct branch of the \C{case} can be selected.
This rule is otherwise straightforward.
$$
\osrule{so}{S_1} E {\mbox{while }e_1\mbox{ loop } e_2\mbox{ pool}}{void}{S_{4}}
	{\begin{array}{l}
	\ossimple{so}{S_1}{E}{e_1}{Bool(true)}{S_2}\\
	\ossimple{so}{S_2}{E}{e_2}{v_2}{S_3}\\
	\ossimple{so}{S_3}{E}{\mbox{while }e_1\mbox{ loop } e_2\mbox{ pool}}{v_3}{S_4}\\
	 \end{array}}{[Loop-True]}
$$

$$
\osrule{so}{S_1} E {\mbox{while }e_1\mbox{ loop } e_2\mbox{ pool}}{void}{S_{2}}
	{\begin{array}{l}
	\ossimple{so}{S_1}{E}{e_1}{Bool(false)}{S_2}\\
	 \end{array}}{[Loop-False]}
$$
There are two rules for \C{while}: one for the case where the predicate is
\C{true} and one for the case where the predicate is \C{false}.  Both cases
are straightforward. The two rules for \C{isvoid} are also straightforward:
$$
\osrule{so}{S_1}{E}{\m{isvoid } e_1}{Bool(true)}{S_2}
	{\ossimple{so}{S_1}{E}{e_1}{void}{S_2}}
	{[IsVoid-True]}
$$

$$
\osrule{so}{S_1}{E}{\m{isvoid } e_1}{Bool(false)}{S_2}
	{\ossimple{so}{S_1}{E}{e_1}{X(\ldots)}{S_2}}
	{[IsVoid-False]}
$$
The remainder of the rules are for the primitive arithmetic, logical, and 
comparison operations except equality.  These are all easy rules.
$$
\osrule{so}{S_1} E {\mbox{not }e_1}{v_1}{S_2}
	{\begin{array}{l}
	\ossimple{so}{S_1}E{e_1}{Bool(b)}{S_2}\\
	v_1 = Bool(\neg b)
	 \end{array}}{[Not]}
$$

$$
\osrule{so}{S_1} E {e_1\ op\  e_2}{v_1}{S_3}
	{\begin{array}{l}
	\ossimple{so}{S_1}E{e_1}{Int(i_1)}{S_2}\\
	\ossimple{so}{S_2}E{e_2}{Int(i_2)}{S_3}\\
	op \in \{ \leq, < \} \\
	v_3 = \left\{\begin{array}{l}
		Bool(true),\mbox{ if $i_1\ op\  i_2$}\\
		Bool(false),\mbox{ otherwise}
		     \end{array}\right.
	 \end{array}}{[Comp]}
$$


$$
\osrule{so}{S_1}{E}{\mbox{\~{}}e_1}{v_1}{S_2}
	{\begin{array}{l}
	\ossimple{so}{S_1}E{e_1}{Int(i_1)}{S_2}\\
	v_1 = Int(-i_1)\\
	 \end{array}}{[Neg]}
$$

$$
\osrule{so}{S_1} E {e_1\ op\ e_2}{v_1}{S_3}
	{\begin{array}{l}
	\ossimple{so}{S_1}E{e_1}{Int(i_1)}{S_2}\\
	\ossimple{so}{S_2}E{e_2}{Int(i_2)}{S_3}\\
	op \in \{ *, +, -, / \}\\
	v_3 = Int(i_1\ op\ i_2)
	 \end{array}}{[Arith]}
$$
Cool {\tt Int}s are 32-bit two's complement signed integers; the arithmetic
operations are defined accordingly.

The notation and rules given above are not powerful enough to
describe how objects are tested for equality, or how runtime
exceptions are handled.  For these cases we resort to an
English description.

Two objects are compared for equality by first comparing their
pointers (addresses). If they are the same, the objects are equal. 
The value {\tt void} is not equal to any object except itself.  
If the two objects are of type \C{String}, \C{Bool}, or \C{Int}, their respective
contents are compared.

In addition, the operational rules do not specify what happens in the
event of a runtime error.  Execution aborts when a runtime error
occurs. The following list specifies all possible runtime errors.
\begin{enumerate}
\item A dispatch (static or dynamic) on {\tt void}.
\item A case on {\tt void}.
\item Execution of a case statement without a matching branch.
\item Division by zero.
\item Substring out of range.
\item Heap overflow.

\end{enumerate}


Finally, the rules given above do not
explain the execution behaviour for dispatches to primitive methods defined in
the \C{Object}, \C{IO}, or \C{String} classes.
Descriptions of these primitive methods are given in Sections~\ref{sec-int}-\ref{sec-bool}.

\section{Acknowledgements}

Cool is based on Sather164, which is itself based on the language
Sather.  Portions of this document were cribbed from the Sather164
manual; in turn, portions of the Sather164 manual are based on Sather
documentation written by Stephen M. Omohundro.

A number people have contributed to the design and implementation of
Cool, including Manuel F\"{a}hndrich, David Gay, Douglas Hauge,
 Megan Jacoby, Tendo Kayiira, Carleton Miyamoto, and Michael Stoddart.

\end{document}

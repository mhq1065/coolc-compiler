\documentstyle[11pt]{article}

% Side margins:
% Actual margin is 1 in + this number
\oddsidemargin -0.25in
\evensidemargin -0.25in

% Text width:
\textwidth 6.9in

% Top margin:
% Actual margin is 1.5 in + this number
\topmargin -.3in

% Text height:
\textheight 8.7in

\begin{document}

\title{Cool Instructor's Guide}
\author{Alex Aiken \\ EECS Department \\ University of California, Berkeley \\
aiken@cs.berkeley.edu}
\date{ }
\maketitle

Cool, the {\em Classroom Object-Oriented Language}, is a small
programming language designed for teaching the basics of
compiler construction to undergraduate computer science majors.  I
have used Cool for the past two years in teaching the compiler course
at Berkeley.  In the process, I've learned that developing a polished
compiler course project is a labor-intensive exercise in software
development.  Having complete that exercise, I am distributing Cool in
the hope that instructors at other institutions can benefit from the
effort and experience of myself, the other developers of Cool, and the
many Berkeley undergraduates who have written Cool compilers
and provided criticism of the project.

Cool is designed to be implemented by a team of 2 or 3 undergraduates
in C++ in a single semester.  At Berkeley, 80-90\% of the student
teams complete the project each semester.  Interesting subsets of the
language can be implemented by individuals in a single quarter, and
even shorter projects are possible.

This Guide is divided into five sections covering installation of the
system, an overview of the contents of the distribution, a brief rationale
of the Cool design, an outline of how I have used Cool as a course project,
and a discussion of possible alternative ways to use Cool.  

\section{Installation}

Cool is being distributed both in binary and source form.  Binary
installation only requires retrieving the {\tt .u} file for a
particular architecture from {\tt http://www.cs.berkeley.edu/$\tilde{
}$aiken/cool} and following the directions in {\tt
README.INSTALL}. Source installation instructions are in the file {\tt
INSTALL} in the top-level directory.  The system is designed to be
very portable and builds with no changes on many systems.  The system
has been built successfully without modification on Sun, HP, and DEC
Alpha, and DECstation workstations, as well as under Linux on PC's
(both pre- and post-ELF).

The Cool WWW site has only binary distributions available.  I am
happy to send the source distribution to instructors, but as I am
still using Cool at Berkeley, I am not making source code for a Cool
compiler publicly available.

\section{Overview}

The Cool binary distribution comes with:
\begin{itemize}

\item {\tt CoolAid}, a reference manual for Cool.  The manual describes
the syntax and semantics of Cool, as well as how to execute Cool programs.
The manual is divided into an informal and a formal part.  The first, informal
part gives an overview of the language, examples, and an English description
of each of the language's constructs.  The second, formal part of the manual
gives precise descriptions of the lexical structure, grammar, type rules,
and operational semantics of Cool.

\item The {\tt Cool Tour}, an overview of the source code provided to students
to assust them in building a compiler.

\item Documentation for GNU
      tools students use to write portions of the project: {\tt flex}, {\tt bison}, and {\tt gmake}.

\item {\tt coolc}, a reference Cool compiler.  {\tt coolc} has been extensively
tested over the last two years; there are currently no known bugs in the 
compiler.

\item The lexer, parser, semantic analysis, and code generation components
of {\tt coolc}.  The Cool compiler project is modular and
students may, for example, combine their lexer and semantic analysis with
{\tt coolc}'s parser and code generator to produce a complete compiler.

\item {\tt spim}, a MIPS simulator distributed by James Larus.  {\tt coolc}
generates MIPS assembly code.

\item A suite of example Cool programs, small and large. 

\item A set of programming assignments, including both handouts and code
skeletons.

\end{itemize}

The source distribution also includes source code for {\tt coolc}, a
a large collection of valid and invalid Cool programs useful for
testing Cool compilers, and Makefiles for installing the system.

\section{Cool}

Cool is a small language, designed for teaching.  Cool is
object-oriented, strongly typed, and has automatic memory management.
These are the essential features of a number of languages, of which
the most recent and probably best known is Java.

Cool syntax is, for the most part, simple and regular.  Cool
deliberately does not ``look'' exactly like any language with which
students are likely to be familiar.  The intention is to encourage
students to think consciously about the semantics of Cool when
implementing their compilers, rather than relying entirely on
intuitions borrowed from other languages.

The lexical structure of Cool is quite straightforward, with the exception
of a few twists added intentionally to make the lexical analysis
assignment more challenging (e.g., nested comments).

As part of the project, students must write a grammar for Cool.  The
grammar given in the manual is a partial solution, but it uses a
number of non-context-free features.  Almost all Cool constructs
except infix arithmetic expressions have both opening and closing
keywords (e.g., {\tt begin \ldots end}).  This design makes for
verbose Cool programs, but it also makes designing, debugging, and extending the
parser significantly easier.  The collection of infix expressions
introduce ambiguities which students must resolve either by writing a
LALR(1) grammar or using precedence declarations.

Cool has a fairly conventional object-oriented type system, in which
(single) inheritance also defines the subtyping relation.  The only
subtle feature in the Cool type system is {\tt SELF\_TYPE}, the type
of the {\tt self} object.  Cool has several different kinds of
identifiers with varying scope rules illustrating common design
choices.

The execution of Cool programs uses a standard object-oriented model
with dynamic dispatch.  All memory management is automatic; {\tt
coolc} has a generational garbage collector.  A couple of surprises
have been engineered into the operational semantics.  In particular,
the {\tt case} expression provides a form of runtime dynamic type
checking that is trickier to implement correctly than it first
appears.  Also, the semantics of {\tt new} turns out to be more
involved than students imagine.

\section{Teaching Cool}

As with most compiler projects, a Cool implementation is most naturally 
built in four stages: lexer, parser, semantic analysis, and finally
code generation.  I add a fifth assignment: writing a Cool program.
Writing a program in Cool before writing a compiler familiarizes students
with the language, the manual, {\tt coolc}, and provides the students
with a substantial test case for their project.  For a 15-week semester 
course,
the time I allot for each assignment is: \\[.2in]
\begin{tabular}{lll}
 & {\em assignment} & {\em weeks} \\
1 & Cool program & 1 \\
2 & lexical analysis & 2 \\
3 & parser & 2.5 \\
4 & semantic analysis & 4 \\
5 & code generation & 4 \\[.2in]
\end{tabular}

A very important aspect of the project is that students are able to
proceed even if they do not complete one of the assignments.
Students can use any of the four components of {\tt coolc} to test the
component they are writing.  Thus, a student writing a parser may test
it by linking with {\tt coolc}'s semantic analysis and code
generation and either the student's or {\tt coolc}'s lexical
analyzer.  Thus, there are no dependencies between the assignments
that penalize a student who fails to finish one of the early components.

Most students find lexical analysis to be the easiest assignment.
However, in my experience it is also the assignment with the lowest
average grade, as it is very easy to make obscure errors using {\tt flex}-like
tools.

The parsing assignment requires the least amount of coding effort.
The assignment still requires a considerable amount of time, however,
because students must learn to use a parser generator and the details
of Cool's representation of abstract syntax trees.

There is a large step up in implementation complexity between the
parsing assignment and the semantic analysis assignment; where the
parser can be written in at most a few hundred lines, the semantic
analyzer generally requires 1000-1500 lines of code.  It is important
to stress to the students that this assignment has more time allocated
to it for a reason, but, inevitably, a few students lulled into a false
sense of security by earlier assignments fail to complete this
assignment. 
 I make the fourth and fifth assignments easier by telling students
not to worry about memory management---they should just allocate as
much space as they like and not worry about deallocating objects that
are no longer in use.  I also emphasize correctness as the most important
goal.  Students are free to use simple, inefficient implementations
so long as the compiler works as specified.

The final assignment is a little larger than the semantic analysis component,
but many students find it easier overall because there is no error checking
(the hard part of semantic analysis is making sure that all erroneous
programs are detected) and because they can compare their compiler's
generated code directly against that of {\tt coolc}. 

Each of the assignments comes with a handout (\LaTeX source is available
in the {\tt cool/handouts} directory) and a {\tt README} file (see
the directories under {\tt cool/assignments}).  The instructions should
be modified with information for individual courses (e.g., how to turn
in assignments).

\section{Alternatives}

There are a number of ways that the {\em Cool} project can be varied
to save time and effort and still be a coherent project.  These
variations inlcude

\begin{itemize}

\item {\em Subset the language.} There are two orthogonal ways to subset
the language.  Removing a language feature will make all phases of the 
compiler easier.  For example, dropping {\tt case} expressions
simplifies all phases, but particularly semantic analysis and code generation.
Particular phases can also be simplified.  For example, dropping
{\tt SELF\_TYPE} makes semantic analysis significantly easier to implement.

\item {\em Eliminate a phase.} Semantic analysis or code generation could
be skipped and the corresponding component from {\tt coolc} always used in
its place.

\item {\em Give bigger skeletons.}  The given assignment skeletons are modest;
more {\tt coolc} source code could be given as a starting point.

\end{itemize}

Another alternative is to use Java as the implementation language for
the semantic analysis and code generation assignments.  It should be
very easy to translate the {\em coolc} source and skeleton code into
Java, and the fact that Java is a safe programming language would make
the project easier for many students.  When development tools (such as
debuggers and a good parser generator) are available for Java, a port
of the entire system to Java would be worthwhile.

Of course, it is also possible (though not likely) to have too much
time in a course, in which case the project can be extended.  I usually
assign an extra-credit optimization phase (the students implement any
optimization of their choice and measure the results) in the last week
of the course.


\end{document}

